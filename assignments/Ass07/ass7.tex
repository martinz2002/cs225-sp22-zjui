%        -----------------------------
%        | Group Assignment Template |
%        -----------------------------
%         Author: Zhong Tiantian
%         Date Created: Feb 21, 2022
%         Date Modified: Feb 22, 2022
%         Version: 2.0
%
% -------------------------------------------

\documentclass[12pt]{article}
\usepackage[scale=0.8]{geometry}
\usepackage{fontspec}   % 启用自定义字体
\usepackage{fancyhdr}
\usepackage{graphicx}   % 支持图像
\usepackage{hyperref}   % 支持引用(书签引用)
\usepackage{amsmath}    % 支持数学
\usepackage{amsfonts}   % 支持数学字体
\usepackage{amssymb}    % 支持数学符号
\usepackage{enumerate}
\usepackage{bm}
\usepackage{mathptmx}
\usepackage[nofillcomment, linesnumbered,ruled,lined,boxed]{algorithm2e}% 支持伪代码
\usepackage{listings}
\usepackage{fontspec}
\usepackage{booktabs}
\usepackage{tikz}
\usetikzlibrary {graphs, graphdrawing}
\usegdlibrary {trees}

% \everymath{\displaystyle}
\setmainfont{Times New Roman}
% \setmonofont{Consolas}
% \setsansfont{Arial}
\lstset{                  %设置代码块
    basicstyle=\small\fontspec{Consolas},% 基本风格
    numbers=left,    % 行号
    numbersep=10pt,  % 行号间隔
    tabsize=4,       % 缩进
    extendedchars=true, % 扩展符号?
    breaklines=true, % 自动换行
    frame=shadowbox,  % 框架左边竖线
    xleftmargin=19pt,% 竖线左边间距
    showspaces=false,% 空格字符加下划线
    showstringspaces=false,% 字符串中的空格加下划线
    showtabs=false,  % 字符串中的tab加下划线
}
\pagestyle{fancy}
\fancyhf{}
\fancyhead[L]{\textsc\COURSECODE \\ \textbf\HWTITLE}
\fancyhead[C]{}
\fancyhead[R]{Last Modified on \\ \today}
\fancyfoot[L]{\AUTHOR}
\fancyfoot[C]{}
\fancyfoot[R]{--\thepage--}

\fancypagestyle{firstPage}{
    \fancyhf{}
    \fancyfoot[R]{--\thepage--}
}

\newcommand{\E}{\text{e}}
\newcommand{\Diff}{\text{d}}

\title{\textsc\COURSECODE \\ \textbf{\HWTITLE}}
\author{\AUTHOR}
\date{\textit{\normalsize Last Modified on \today}}

\newenvironment{questions}{\begin{enumerate}[Ex. 1]}{\end{enumerate}}
\newenvironment{parts}{\begin{enumerate}[(i)]}{\end{enumerate}}
\newcommand{\question}[1]{\newpage \item \textsc{\textbf{Our Answer.}} (This part is written by #1)\newline}
\renewcommand\part{\item}
\newcommand\makeMyTitle{\maketitle
                        % Please add the following required packages to your document preamble:
% \usepackage{booktabs}
\begin{table}[htp]
    \centering
    \begin{tabular}{@{}clc@{}}
        \toprule
        \multicolumn{2}{c}{\textbf{Members}}                                                               \\ \midrule\midrule
        \multicolumn{1}{c|}{\textbf{Name}}  & \textbf{Student ID} \\ \midrule
        \multicolumn{1}{c|}{Li Rong}        & 3200110523          \\
        \multicolumn{1}{c|}{Zhong Tiantian} & 3200110643          \\
        \multicolumn{1}{c|}{Zhou Ruidi}     & 3200111303          \\
        \multicolumn{1}{c|}{Jiang Wenhan}   & 3200111016          \\ \bottomrule
    \end{tabular}
\end{table} 
                        \vspace{5cm}
                        \centering\textsc{\Large{Please turn over for our answers.}}\normalsize}

\renewcommand{\baselinestretch}{1.2}
% \setlength{\parskip}{0ex}



\def\HWTITLE{Assignment 7}
\def\COURSECODE{CS 225: Data Structures}
\def\AUTHOR{Group D1}


\begin{document}
\makeMyTitle

Exercise 4 is written by LI Rong.


\thispagestyle{empty}

\begin{questions}
    \question{ZHONG Tiantian} % ex.1
    \begin{parts}
        \part
        Since in an AVL tree the left child is always smaller than its parent, and the right bigger, we can perform a recursive traverse to do this job. Note that the minimum should always exist as a left child of a node, and the maximum as a right child.

        Our algorithm is described in Function \ref*{proc-min} and Function \ref*{proc-max}.

        \begin{function}
            \caption{MINIMUM(node)}
            \label{proc-min}
            \SetKwData{Node}{node}
            \SetKwData{Left}{leftChild}
            \SetKwData{Right}{rightChild}
            \SetKwData{Null}{Null}
            \SetKw{Call}{call}

            \If{$\Left = \Null$}{
                \Return{\Node}
            }

            \Return{\MINIMUM{\Left}}
        \end{function}

        \begin{function}
            \caption{MAXIMUM(node)}
            \label{proc-max}
            \SetKwData{Node}{node}
            \SetKwData{Left}{leftChild}
            \SetKwData{Right}{rightChild}
            \SetKwData{Null}{Null}
            \SetKw{Call}{call}

            \If{$\Right = \Null$}{
                \Return{\Node}
            }

            \Return{\MAXIMUM{\Right}}
        \end{function}

        \part
        Define a function CheckAVL, which computes the depth of the tree with root \texttt{node} and returns \texttt{True} if the tree rooted with \texttt{node} is an AVL tree.

        \paragraph{Labelling the depth.} By running function FindDepth, we label each node with their depth, assuming the root node has depth 0. Then iterating each node recursively by Depth-First Search and find the depth for all the nodes.

        \begin{procedure}
            \caption{FindDepth(node, parentDepth)}
            \label{func-finddepth}
            \SetKwData{Node}{node}
            \SetKwData{Root}{root}
            \SetKwData{Queue}{queue}
            \SetKwArray{Left}{leftChild}
            \SetKwArray{Right}{rightChild}
            \SetKwArray{ParentDepth}{parentDepth}
            \SetKwData{MaxDepth}{maxDepth}
            \SetKwData{Null}{Null}
            \SetKwArray{Depth}{depth}
            \SetKwFunction{ChkAVL}{CheckAVL}
            \SetKwFunction{DFS}{FindDepth}

            \tcc{Find depth of each node using Depth-Frist Search}

            $\Depth{\Node}\gets \ParentDepth + 1$

            \If{$\Depth{\Node} > \MaxDepth$}{$\MaxDepth\gets \Depth{\Node}$}

            \If{\Node is a leaf child}
            {
                \Exit
            }

            \If{\Node has a left child}{
                \Call \DFS{\Left{\Node},\Depth{\Node}}
            }

            \If{\Node has a right child}{
                \Call \DFS{\Right{\Node},\Depth{\Node}}
            }

        \end{procedure}

        \paragraph{Compare the depth.} By running function CompareDepth, we use Breadth-First Search to traverse the tree. When encountered with a leaf node, the function compares its depth with the maximum depth, i.e. the height of the tree. If the difference is larger than 1, the tree then can be determined not satisfying AVL tree properties.

        See Algorithm \ref*{algo-depth}.

        \begin{algorithm}
            \caption{AVL-Check}
            \label{algo-depth}
            \SetKwData{Node}{node}
            \SetKwData{Root}{root}
            \SetKwData{Queue}{queue}
            \SetKwArray{Left}{leftChild}
            \SetKwArray{Right}{rightChild}
            \SetKwArray{ParentDepth}{parentDepth}
            \SetKwData{MaxDepth}{maxDepth}
            \SetKwData{Null}{Null}
            \SetKwArray{Depth}{depth}
            \SetKwFunction{ChkAVL}{CheckAVL}
            \SetKwFunction{DFS}{FindDepth}

            \If{\Root does not exist, i.e. an empty tree}{
                \Return{False}
            }

            $\Depth{\Root}\gets 1$

            \Call{\DFS{\Root,\Depth{\Root}}}

            append \Root to \Queue

            \While{$\Queue\neq\emptyset$}{
                $\Node \gets$ the front element in \Queue

                pop \Queue's front element

                \If{\Node is a leaf node}{
                    \If{$\Depth{\Node} - \MaxDepth \geq 2$}{\Return{\False}}
                }
                \Else{append \Node's children to \Queue}

            }

            \Return{\True}

        \end{algorithm}


    \end{parts}

    \question{JIANG Wenhan, ZHONG Tiantian}
    \begin{parts}
        \part % 2.i
        Applying a search sequence from root to one leaf node (fixed directoin), then we return to the root and begin another search from root to another leaf node. Repeat such operation until all leaf nodes are traversed.

        A structured description is provided in Algorithm \ref*{algo-dfs}.

        \begin{algorithm}
            \caption{Depth-Frist Search}
            \label{algo-dfs}
            \SetKwData{Node}{node}
            \SetKwArray{Left}{left\_child}
            \SetKwArray{Right}{right\_child}
            \SetKwArray{Key}{key}

            \SetKwData{List}{list\_of\_elements}
            \SetKwData{Root}{root}
            \SetKwFunction{DFS}{DepthFirstSearch}

            \tcp{Initialize an empty list}

            $\List\gets []$

            \BlankLine

            \Proc{\DFS{\Node}}{

                append \Node to \List

                \If{\Node is a leaf child}
                {
                    \Exit
                }

                \If{\Node has a left child}{
                    append \Key{\Node} to \List

                    \Call \DFS{\Left{\Node}}
                }

                \If{\Node has a right child}{
                    append \Key{\Node} to \List

                    \Call \DFS{\Right{\Node}}
                }
            }

            \tcc{Main}

            \Begin{

                \Call \DFS{\Root}

                \Return{\List}
            }


        \end{algorithm}

        \part % 2.2
        Applying a search sequence firstly among all child nodes of root node and add the children to the queue. The queue records the nodes to visit in the following steps.

        Retrieve the top element (denoting it with $e$) in the queue, and \textbf{append} all the child nodes of $e$ to the queue. Popping $e$, and repeat the operation until the queue is empty.

        A structured description is provided in Algorithm \ref*{algo-bfs}.

        \begin{algorithm}[h]
            \caption{Breadth-First Search}
            \label{algo-bfs}
            \SetKwData{Queue}{visit\_queue}
            \SetKwData{Root}{root}
            \SetKwData{Node}{node}
            \SetKwData{Result}{result\_list}

            append \Root to \Queue

            \While{$\Queue\neq\emptyset$}{
                $\Node \gets$ the front element in \Queue

                pop \Queue's front element

                append \Node's children (if any) to \Queue
            }

            \Return{\Result}


        \end{algorithm}
        \pagebreak
        \part % 2.3
        Create a queue recording the element in AVL tree in an ascending sequence. To get such a queue, we can simply call a BFS or a DFS. After this step the length of queue can be used to determine the median of the tree.

        Note that since we are performing an inorder transversal, the elements in result list must be ordered ascending.



        \begin{algorithm}[h]
            \caption{Find Median}
            \label{algo-median}
            \SetKwArray{Queue}{visit\_queue}
            \SetKwData{Len}{length}
            \SetKwData{Root}{root}
            \SetKwData{Node}{node}
            \SetKwData{Result}{result\_list}
            \SetKwData{MedIndex}{median\_index}
            \SetKwFunction{Search}{Search}
            $\Queue\gets$ []

            \Call\Search{\Root} \tcp{whatever BreadthFirstSearch() or DepthFirstSearch()}

            \If{\Len is an odd number}{
                \Return{$\Queue{\Len / 2 + 1}$}
            }
            \Else{
                \Return{$1/2 \cdot \left( \Queue{\Len / 2} + \Queue{\Len / 2 + 1} \right)$}
            }
        \end{algorithm}

        \part %iv
        Still insist on Depth-First Search. But we will make the ``digging'' process stop when reaching the boundaries.

        You may refer to Algorithm \ref*{algo-range}.



    \end{parts}

    \begin{algorithm}
        \caption{Range}
        \label{algo-range}
        \SetKwData{Node}{node}
        \SetKwArray{Left}{left\_child}
        \SetKwArray{Right}{right\_child}
        \SetKwArray{Key}{key}
        \SetKwData{ShouldAppend}{shouldAppend}
        \SetKwData{List}{list\_of\_elements}
        \SetKwData{Root}{root}
        \SetKwFunction{DFS}{DepthFirstSearch}

        \tcp{Initialize an empty list}

        $\List\gets []$

        \BlankLine

        \Proc{\DFS{\Node, \ShouldAppend}}{

            \If{\Node does not exist}
            {
                \Exit
            }

            \If{$\ShouldAppend = \False \wedge \Key{\Node} > x$}{
                $\ShouldAppend \gets \True$
            }

            \If{$\ShouldAppend = \True \wedge \Key{\Node} < x$}{
                \Exit
            }

            \Call \DFS{\Left{\Node},\ShouldAppend}

            \If{$\ShouldAppend = \True$}{
                append \Key{\Node} to \List
            }

            \Call \DFS{\Right{\Node},\ShouldAppend}
        }

        \tcc{Main}

        \Begin{

            \Call \DFS{\Root, \False}

            \Return{\List}
        }

    \end{algorithm}

    \question{ZHOU Ruidi}
    \begin{parts}
        \part\label{part}
        We consider the case for a single layer and the depth of a tree.\

        \paragraph{A Single Layer.} For each level, we use binary search with $(b-1)$ nodes (since a root/vertex has at most $b$ children); thus the comparison times for a single level is
        \[ \log(b-1) < \log b\]

        \paragraph{Depth of the Tree.} Because the minimum number of child nodes of a vertex is $a$, and all the leaf nodes lies in the same layer, the depth of the tree, $H$, is\[ H\leq \log_a(n-1) \leq \log_a(n-1)+(2-\log_a 2)\]

        Thus the total times is at most
        \begin{align*}
            T & \leq \log(b-1) \times H                                                    \\
              & \leq \log(b-1) \times \log_a(n-1)                                          \\
              & \leq \left\lceil \log b\right\rceil \left((\log_a(n-1)+(2-\log_a 2)\right)
        \end{align*}
        Hence proved.

        \part
        From Ex. \ref{part} we have the total number of comparison is
        \[
            T=\left\lceil \log b\right\rceil \left((\log_a(n-1)+(2-\log_a 2)\right)
        \]

        By the $O$-notation,
        \begin{equation}
            \label{eq1}
            2\log b \in O(\log b)
        \end{equation}

        At the same time, we know that because $a$ is a constant,
        \begin{equation}
            \label{eq2}
            \log_a \frac{n-1}{2} \in O(\log n)
        \end{equation}

        Plus $b$ is a constant, it's obvious that
        \begin{equation*}
            \log b \cdot \log_a \frac{n-1}{2} \in O(\log n)
        \end{equation*}

        Hence adding equation (\ref*{eq1}) and (\ref*{eq2}) will produce the result
        \begin{equation*}
            O(\log b) + O(\log n)
        \end{equation*}

    \end{parts}
\end{questions}

\end{document}