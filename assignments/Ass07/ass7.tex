%        -----------------------------
%        | Group Assignment Template |
%        -----------------------------
%         Author: Zhong Tiantian
%         Date Created: Feb 21, 2022
%         Date Modified: Feb 22, 2022
%         Version: 2.0
%
% -------------------------------------------

\documentclass[12pt]{article}
\usepackage[scale=0.8]{geometry}
\usepackage{fontspec}   % 启用自定义字体
\usepackage{fancyhdr}
\usepackage{graphicx}   % 支持图像
\usepackage{hyperref}   % 支持引用(书签引用)
\usepackage{amsmath}    % 支持数学
\usepackage{amsfonts}   % 支持数学字体
\usepackage{amssymb}    % 支持数学符号
\usepackage{enumerate}
\usepackage{bm}
\usepackage{mathptmx}
\usepackage[nofillcomment, linesnumbered,ruled,lined,boxed]{algorithm2e}% 支持伪代码
\usepackage{listings}
\usepackage{fontspec}
\usepackage{booktabs}
\usepackage{tikz}
\usetikzlibrary {graphs, graphdrawing}
\usegdlibrary {trees}

% \everymath{\displaystyle}
\setmainfont{Times New Roman}
% \setmonofont{Consolas}
% \setsansfont{Arial}
\lstset{                  %设置代码块
    basicstyle=\small\fontspec{Consolas},% 基本风格
    numbers=left,    % 行号
    numbersep=10pt,  % 行号间隔
    tabsize=4,       % 缩进
    extendedchars=true, % 扩展符号?
    breaklines=true, % 自动换行
    frame=shadowbox,  % 框架左边竖线
    xleftmargin=19pt,% 竖线左边间距
    showspaces=false,% 空格字符加下划线
    showstringspaces=false,% 字符串中的空格加下划线
    showtabs=false,  % 字符串中的tab加下划线
}
\pagestyle{fancy}
\fancyhf{}
\fancyhead[L]{\textsc\COURSECODE \\ \textbf\HWTITLE}
\fancyhead[C]{}
\fancyhead[R]{Last Modified on \\ \today}
\fancyfoot[L]{\AUTHOR}
\fancyfoot[C]{}
\fancyfoot[R]{--\thepage--}

\fancypagestyle{firstPage}{
    \fancyhf{}
    \fancyfoot[R]{--\thepage--}
}

\newcommand{\E}{\text{e}}
\newcommand{\Diff}{\text{d}}

\title{\textsc\COURSECODE \\ \textbf{\HWTITLE}}
\author{\AUTHOR}
\date{\textit{\normalsize Last Modified on \today}}

\newenvironment{questions}{\begin{enumerate}[Ex. 1]}{\end{enumerate}}
\newenvironment{parts}{\begin{enumerate}[(i)]}{\end{enumerate}}
\newcommand{\question}[1]{\newpage \item \textsc{\textbf{Our Answer.}} (This part is written by #1)\newline}
\renewcommand\part{\item}
\newcommand\makeMyTitle{\maketitle
                        % Please add the following required packages to your document preamble:
% \usepackage{booktabs}
\begin{table}[htp]
    \centering
    \begin{tabular}{@{}clc@{}}
        \toprule
        \multicolumn{2}{c}{\textbf{Members}}                                                               \\ \midrule\midrule
        \multicolumn{1}{c|}{\textbf{Name}}  & \textbf{Student ID} \\ \midrule
        \multicolumn{1}{c|}{Li Rong}        & 3200110523          \\
        \multicolumn{1}{c|}{Zhong Tiantian} & 3200110643          \\
        \multicolumn{1}{c|}{Zhou Ruidi}     & 3200111303          \\
        \multicolumn{1}{c|}{Jiang Wenhan}   & 3200111016          \\ \bottomrule
    \end{tabular}
\end{table} 
                        \vspace{5cm}
                        \centering\textsc{\Large{Please turn over for our answers.}}\normalsize}

\renewcommand{\baselinestretch}{1.2}
% \setlength{\parskip}{0ex}



\def\HWTITLE{Assignment 7}
\def\COURSECODE{CS 225: Data Structures}
\def\AUTHOR{Group D1}


\begin{document}
\makeMyTitle

Exercise 4 is written by LI Rong.


\thispagestyle{firstPage}

\begin{questions}
    \question{ZHONG Tiantian} % ex.1
    \begin{parts}
        \part
        Since in an AVL tree the left child is always smaller than its parent, and the right bigger, we can perform a DFS to do this job. Our algorithm is described in Function \ref*{proc-min} and \ref*{proc-max}.

        \begin{function}
            \caption{MINIMUM(node)}
            \label{proc-min}
            \SetKwData{Node}{node}
            \SetKwData{Left}{leftChild}
            \SetKwData{Right}{rightChild}
            \SetKwData{Null}{Null}
            \SetKw{Call}{call}

            \If{$\Left = \Null$}{
                \Return{\Node}
            }

            \Return{\MINIMUM{\Left}}
        \end{function}

        \begin{function}
            \caption{MAXIMUM(node)}
            \label{proc-max}
            \SetKwData{Node}{node}
            \SetKwData{Left}{leftChild}
            \SetKwData{Right}{rightChild}
            \SetKwData{Null}{Null}
            \SetKw{Call}{call}

            \If{$\Right = \Null$}{
                \Return{\Node}
            }

            \Return{\MAXIMUM{\Right}}
        \end{function}

        \part
        Define a function CheckAVL, which computes the depth of the tree with root \texttt{node}. See Function \ref*{fun-depth}.

        \begin{function}[h]
            \caption{CheckAVL(node,depthOfParent,maxDepth)}
            \label{fun-depth}
            \SetKwData{ParentDepth}{depthOfParent}
            \SetKwData{MaxDepth}{maxDepth}
            \SetKwData{Node}{node}
            \SetKwData{Left}{leftChild}
            \SetKwData{Right}{rightChild}
            \SetKwData{Null}{Null}
            \SetKwData{Depth}{depth}
            \SetKwData{IsAVL}{isAVL}
            \SetKw{Call}{call}


        \end{function}
    \end{parts}

    \question{JIANG Wenhan, ZHONG Tiantian}
    \begin{parts}
        \part % 2.i
        Applying a search sequence from root to one leaf node (fixed directoin), then we return to the root and begin another search from root to another leaf node. Repeat such operation until all leaf nodes are traversed.

        A structured description is provided in Algorithm \ref*{algo-dfs}.

        \begin{algorithm}
            % \SetKwFunction{FRecurs}{FnRecursive}
            % \Fn(\tcc*[h]{algorithm as a recursive function}){\FRecurs{some args}}{
            %     \KwData{Some input data\\these inputs can be displayed on several lines and one
            %         input can be wider than line's width.}
            %     \KwResult{Same for output data}
            %     \tcc{this is a comment to tell you that we will now really start code}
            %     \If(\tcc*[h]{a simple if but with a comment on the same line}){this is true}{
            %         we do that, else nothing\;
            %         \tcc{we will include other if so you can see this is possible}
            %         \eIf{we agree that}{
            %             we do that\;
            %         }{
            %             else we will do a more complicated if using else if\;
            %             \uIf{this first condition is true}{
            %                 we do that\;
            %             }
            %             \uElseIf{this other condition is true}{
            %                 this is done\tcc*[r]{else if}
            %             }
            %             \Else{
            %                 in other case, we do this\tcc*[r]{else}
            %             }
            %         }
            %     }
            %     \tcc{now loops}
            %     \For{\forcond}{
            %         a for loop\;
            %     }
            %     \While{$i<n$}{
            %         a while loop including a repeat--until loop\;
            %         \Repeat{this end condition}{
            %             do this things\;
            %         }
            %     }
            %     They are many other possibilities and customization possible that you have to
            %     discover by reading the documentation.
            % }
            \caption{Depth-Frist Search}
            \label{algo-dfs}
            \SetKwData{Node}{node}
            \SetKwData{Left}{leftChild}
            \SetKwData{Right}{rightChild}
            \SetKwData{Key}{key}
            \SetKwData{Null}{null}
            \SetKwData{List}{ListOfElements}
            \SetKwData{Root}{root}
            \SetKwFunction{DFS}{DepthFirstSearch}


            $\List\gets []$ \tcc{Initialize an empty list}

            \Proc{\DFS{\Node}}{
                \If{\Node is a leaf child}
                {
                    append \Key to \List

                    \Exit
                }

                \If{\Node has a left child}{
                    append \Key to \List

                    \DFS{\Left}
                }

                \If{\Node has a right child}{
                    append \Key to \List

                    \DFS{\Left}
                }
            }

            \tcc{Main}
            \Begin{
                \Call \DFS{\Root}

                \Return{\List}
            }

        \end{algorithm}

        \part % 2.2
        Applying a search sequence firstly among all child nodes of root node and add the children to the queue. The queue records the nodes to visit in the following steps.

        Retrieve the top element (denoting it with $e$) in the queue, and \textbf{append} all the child nodes of $e$ to the queue. Popping $e$, and repeat the operation until the queue is empty.

        A structured description is provided in Algorithm \ref*{algo-bfs}.

        \begin{algorithm}
            \caption{Breath-First Search}
            \label{algo-bfs}
            \SetKwData{Queue}{visit\_queue}
            \SetKwData{Root}{root}
            \SetKwData{Node}{node}
            \SetKwData{Result}{result\_list}

            append \Root to \Queue

            \While{$\Queue\neq\emptyset$}{
                $\Node \gets$ the front element in \Queue

                pop \Queue's front element

                append \Node's children (if any) to \Queue
            }

            \Return{\Result}
        \end{algorithm}

        \part % 2.3
        Create a queue recording the element in AVL tree in an ascending sequence. We can achieve such recording through following method: ------------------------.Then we seek for the median value in the sequence and return it.

        \begin{algorithm}
            \caption{Find Median}
            \label{algo-median}
        \end{algorithm}

        \part
        hello world
        % \begin{figure}[h]
        %     \centering

        %     \label{fig-FakeTree}
        %     \tikz [tree layout, sibling distance=8mm, mark/.style={fill=violet, text=white}, mark/.default=, markmin/.style={fill=red, text=white}, markmin/.default=]
        %     \graph [nodes={circle, draw, inner sep=1.5pt}]{ [fresh nodes]

        %     1[markmin] <- {{2 [mark]} <- {2 <- {2, 5}, 3 <- {3, 8}}, 1 <- {4[mark] <- {4, 6}, 1 <- {3[mark], 1}}}

        %     };
        %     \caption{An example structure for the algorithm.}
        % \end{figure}

        hello


    \end{parts}

    \question{ZHOU Ruidi}
    \begin{parts}
        \part\label{part}
        We consider the case for a single layer and the depth of a tree.\

        \paragraph{A Single Layer.} For each level, we use binary search with $(b-1)$ nodes (since a root/vertex has at most $b$ children); thus the comparison times for a single level is
        \[ \log(b-1) < \log b\]

        \paragraph{Depth of the Tree.} Because the minimum number of child nodes of a vertex is $a$, and all the leaf nodes lies in the same layer, the depth of the tree, $H$, is\[ H\leq \log_a(n-1) \leq \log_a(n-1)+(2-\log_a 2)\]

        Thus the total times is at most
        \begin{align*}
            T & \leq \log(b-1) \times H                                                    \\
              & \leq \log(b-1) \times \log_a(n-1)                                          \\
              & \leq \left\lceil \log b\right\rceil \left((\log_a(n-1)+(2-\log_a 2)\right)
        \end{align*}
        Hence proved.

        \part
        From Ex. \ref{part} we have the total number of comparison is
        \[
            T=\left\lceil \log b\right\rceil \left((\log_a(n-1)+(2-\log_a 2)\right)
        \]

        By the $O$-notation,
        \begin{equation}
            \label{eq1}
            2\log b \in O(\log b)
        \end{equation}

        At the same time, we know that because $a$ is a constant,
        \begin{equation}
            \label{eq2}
            \log_a \frac{n-1}{2} \in O(\log n)
        \end{equation}

        Plus $b$ is a constant, it's obvious that
        \begin{equation*}
            \log b \cdot \log_a \frac{n-1}{2} \in O(\log n)
        \end{equation*}

        Hence adding equation (\ref*{eq1}) and (\ref*{eq2}) will produce the result
        \begin{equation*}
            O(\log b) + O(\log n)
        \end{equation*}

    \end{parts}
\end{questions}

\end{document}