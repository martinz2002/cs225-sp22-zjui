% -------------------------------------------
%          请使用XeTeX, XeLaTeX编译
%        -----------------------------
%        | Group Assignment Template |
%        -----------------------------
%         Author: Zhong Tiantian
%         Date Created: Feb 21, 2022
%         Date Modified: Feb 22, 2022
%         
% -------------------------------------------
%
% 关于 LaTeX 教程,请参阅 https://liam.page/2014/09/08/latex-introduction/
%

\documentclass[answers]{exam}
\usepackage{ctex}   % 引入中文支持
\usepackage{graphicx}   % 支持图像
\usepackage{hyperref}   % 支持引用(书签引用)
\usepackage{amsmath}    % 支持数学
\usepackage{amsfonts}   % 支持数学字体
\usepackage{amssymb}    % 支持数学符号
\usepackage{algorithmicx}   % 支持伪代码
% --- set math font ---
\usepackage{mathptm}   % 设置数学公式字体为 Adobe Times
\usepackage{bm}         % 在数学公式中启用粗体
% ---------------------
\usepackage{booktabs}   % 启用书签
% \usepackage[scale=0.8]{geometry}    % 设置页边距
\usepackage{fontspec}   % 启用自定义字体



\CTEXoptions[today=old] % 设置英文日期格式
\unframedsolutions  % 设置Solution不带边框
\setmainfont{Inter} % 设置正文字体为 Inter

% ------ 页面设置 ------

\setlength\linefillheight{.5in} % 行间距
\pagestyle{headandfoot} % 引入页眉和页脚
\firstpageheadrule  % 首页页眉横线
\firstpagefootrule
\runningheadrule    %正文页页眉横线
\runningfootrule
\firstpagefooter{\LARGE{D1}}{}{--\thepage--}    % 首页页脚
\runningfooter{\LARGE{D1}}{}{--\thepage--}  % 正文页页脚

% ------ !!!!填写作业标题!!!!
\firstpageheader{ZJUI CS225 D1}{\huge{\textbf{Homework 1}}\\ \small{Li Rong, Zhou Ruidi, Zhong Tiantian, Jiang Wenhan}}{\today}   % 首页页眉{左}{中}{右}
\runningheader{Homework 1}{}{\today}  % 正文页眉
% ------ !!!!填写作业标题!!!!

\begin{document}
\begin{questions}
    \question
    \begin{solution}
        % 在这里写第一题的解答。
        Solution to 1.
    \end{solution}

    \question
    \begin{parts}
        \part
        \begin{solution}
            % 第二题(i)的解答
            Solution to 2(i)
        \end{solution}
        \part
        \begin{solution}
            % 第二题(ii)的解答
            Solution to 2(ii)
        \end{solution}

        \part
        \begin{solution}
            % 第二题(iii)的解答
            Solution to 2(iii)
        \end{solution}
    \end{parts}

    \question
    \begin{solution}
        % 在这里写第三题的解答。
        Solution to 3
    \end{solution}

    \question
    \begin{solution}
        Please refer to the code bundle submitted.
    \end{solution}
\end{questions}
\end{document}