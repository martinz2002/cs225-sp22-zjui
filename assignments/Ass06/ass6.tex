%        -----------------------------
%        | Group Assignment Template |
%        -----------------------------
%         Author: Zhong Tiantian
%         Date Created: Feb 21, 2022
%         Date Modified: Feb 22, 2022
%         Version: 2.0
%
% -------------------------------------------

\documentclass[12pt]{article}
\usepackage[scale=0.8]{geometry}
\usepackage{fontspec}   % 启用自定义字体
\usepackage{fancyhdr}
\usepackage{graphicx}   % 支持图像
\usepackage{hyperref}   % 支持引用(书签引用)
\usepackage{amsmath}    % 支持数学
\usepackage{amsfonts}   % 支持数学字体
\usepackage{amssymb}    % 支持数学符号
\usepackage{enumerate}
\usepackage{bm}
\usepackage{mathptmx}
\usepackage[nofillcomment, linesnumbered,ruled,lined,boxed]{algorithm2e}% 支持伪代码
\usepackage{listings}
\usepackage{fontspec}
\usepackage{booktabs}
\usepackage{tikz}
\usetikzlibrary {graphs, graphdrawing}
\usegdlibrary {trees}

% \everymath{\displaystyle}
\setmainfont{Times New Roman}
% \setmonofont{Consolas}
% \setsansfont{Arial}
\lstset{                  %设置代码块
    basicstyle=\small\fontspec{Consolas},% 基本风格
    numbers=left,    % 行号
    numbersep=10pt,  % 行号间隔
    tabsize=4,       % 缩进
    extendedchars=true, % 扩展符号?
    breaklines=true, % 自动换行
    frame=shadowbox,  % 框架左边竖线
    xleftmargin=19pt,% 竖线左边间距
    showspaces=false,% 空格字符加下划线
    showstringspaces=false,% 字符串中的空格加下划线
    showtabs=false,  % 字符串中的tab加下划线
}
\pagestyle{fancy}
\fancyhf{}
\fancyhead[L]{\textsc\COURSECODE \\ \textbf\HWTITLE}
\fancyhead[C]{}
\fancyhead[R]{Last Modified on \\ \today}
\fancyfoot[L]{\AUTHOR}
\fancyfoot[C]{}
\fancyfoot[R]{--\thepage--}

\fancypagestyle{firstPage}{
    \fancyhf{}
    \fancyfoot[R]{--\thepage--}
}

\newcommand{\E}{\text{e}}
\newcommand{\Diff}{\text{d}}

\title{\textsc\COURSECODE \\ \textbf{\HWTITLE}}
\author{\AUTHOR}
\date{\textit{\normalsize Last Modified on \today}}

\newenvironment{questions}{\begin{enumerate}[Ex. 1]}{\end{enumerate}}
\newenvironment{parts}{\begin{enumerate}[(i)]}{\end{enumerate}}
\newcommand{\question}[1]{\newpage \item \textsc{\textbf{Our Answer.}} (This part is written by #1)\newline}
\renewcommand\part{\item}
\newcommand\makeMyTitle{\maketitle
                        % Please add the following required packages to your document preamble:
% \usepackage{booktabs}
\begin{table}[htp]
    \centering
    \begin{tabular}{@{}clc@{}}
        \toprule
        \multicolumn{2}{c}{\textbf{Members}}                                                               \\ \midrule\midrule
        \multicolumn{1}{c|}{\textbf{Name}}  & \textbf{Student ID} \\ \midrule
        \multicolumn{1}{c|}{Li Rong}        & 3200110523          \\
        \multicolumn{1}{c|}{Zhong Tiantian} & 3200110643          \\
        \multicolumn{1}{c|}{Zhou Ruidi}     & 3200111303          \\
        \multicolumn{1}{c|}{Jiang Wenhan}   & 3200111016          \\ \bottomrule
    \end{tabular}
\end{table} 
                        \vspace{5cm}
                        \centering\textsc{\Large{Please turn over for our answers.}}\normalsize}

\renewcommand{\baselinestretch}{1.2}
% \setlength{\parskip}{0ex}



\def\HWTITLE{Homework 6}
\def\COURSECODE{CS 225: Data Structures}
\def\AUTHOR{Group D1}


\begin{document}
\makeMyTitle

Exercise 4 is written by Li Rong \& Jiang Wenhan.

\thispagestyle{firstPage}

\begin{questions}



    \question{Zhou Ruidi}
    Let's define $P(X)$, which satisfies that when inserting $x$ elements into the hash table, the probability of each bucket to be empty is $1-P(X=x)$

    \paragraph{Initial condition.} When inserting the first element, the probability of each bucket to be non-empty is
    \[P(X=1)=\frac{1}{m}\]

    \paragraph{Induction steps.} When inserting $n$-th element $e_n$, there are two possible case:

    \begin{enumerate}
        \item $e_n$ belongs to a non-empty bucket. The probability that such case happens is $P(X=n-1)$. The probability of choosing a non-empty bucket from all the buckets is also \[P(X=n-1)\]
        
        Therefore, the total probability contribution of this case is \[P_\text{non-empty}(X=n)=P(X=n-1)\times P(X=n-1) = P^2(X=n-1)\]
        
        \item $e_n$ belongs to an empty bucket. The probability that such case happens is $1-P(X=n-1)$. The probability of choosing a bucket from all non-empty bucket and one (randomly chosen) empty bucket in this case is $P(X=n-1) + 1/m$. Therefore, the total probability contribution of this case is \[P_\text{empty}(X=n)=\left[ 1-P(X=n-1) \right]\times \left[P(X=n-1) + \frac{1}{m}\right]\]
    \end{enumerate}

    Therefore, the probability that each bucket to be non-empty after inserting $n$ elements is \[ P(X=n) = P^2(X=n-1) + (1-P(X=n-1))\times \left[ \frac{1}{m}\cdot P(X=n-1)\right]\]

    Solve this induction equation and we obtain \[P(X=n)=1-\left(1-\frac{1}{m}\right) ^n\]
    
    Therefore, the probability of each bucket to be empty is \[1-P(X=n)=\left( 1-\frac{1}{m} \right) ^n\]
    
    and thus the expectation is \[E(\text{empty}, X=n)=m\cdot(1-P(X=n))=m\cdot \left( 1-\frac{1}{m} \right) ^n\]


    \question{Jiang Wenhan}
    \begin{parts}
        \part
        To calculate the total payment for one customer, we only need to traverse all triples and add up all prices whenever the customer ID fits the one we need. So the time complexity is $O(n)$. ($n$ refers to the number of triples)

        \part
        Assume the size of represented array is $n$, and the size of stored value is $m$. For hashing with chaining,  we assume the hashing is perfect where no empty entry exist. We need to scan all entries to get all return values. So the running time is $O(m)$. For hashing with linear probing, there exist empty entries. But we can not make assertion for each entry whether it is empty or not. So we have to scan all entries, which takes $O(n)$.
    \end{parts}

    \question{Zhong Tiantian}
    \begin{parts}
        \part
        Focusing on condition $h_j(e) \leq M-1$. Assuming we have an element $e$. Every time if the calculated hash value is larger than $M$, then increase the size of the table by \[ M'\gets M + m\cdot 2^j\]

        Obviously, for current $j$ we must have \[ M\leq h_{j-1}(e) < h_{j}(e)\leq M' \] in order to satisfy the properties of the hash function.

        With the assumptions above and suppose we are using the $j$-function $h_j$:

        \paragraph{Hash value 1.}
        It's possible to calculate the current hash value with $j$-th hash function $h_j(e)$. Check if the ``bucket'' for $h_j(e)$ has been full (i.e. the length of ``bucket'' exceeds limit $l$); if not, put $e$ to the corresponding position of $h_j(e)$. This is the first hash value.

        \paragraph{Hash value 2.}
        If bucket with key $h_j(e)$ has been full, calculate the next hash function $h_{j'}(e)$ with $j' = j+1$. From the assumption we made at the beginning, $h_{j'}(e)>M$ must hold. Thus the bucket for $h_{j'}(e)>M$ must be empty, which means we can safely put the element into this position. After that an increase to the table size by \[ M\gets M + m\cdot 2^j\] can maintain our assumption.

        And this also proves that two hash value for an element $e$ suffices.

        \part
        Define
        \begin{itemize}
            \item a dynamically-sized hash table with $M$ elements, where the size of the table is controlled by $M$.
            \item the hash function as \texttt{Hash($j$, $e$)} which is an implementation of $h_j(e)$.
            \item lists \texttt{HashList[$i$]} as the list of element whose hash value equals to $i$.
                  % \item function \texttt{isEmpty()} to check if a list is empty. 
                  % \item function \texttt{isFull()} to check if a list is full
        \end{itemize}

        The algorithm \texttt{rehashing} looks like Algorithm \ref{algo-rehashing1}.

        \part
        Rewrite Algorithm \ref{algo-rehashing1} to obtain Algorithm \ref{algo-rehashing2}, where we add a checking process to ensure the corresponding list of an element is only splitted if the original list is full, to avoid unnecessary splitting.

    \end{parts}
\end{questions}

\begin{algorithm}[h]
    \label{algo-rehashing1}
    \caption{Rehashing v1}
    \SetKwInOut{Input}{Input}
    \SetKwInOut{Output}{Output}
    \SetKwFunction{Hash}{Hash}
    % \SetKwFunction{Rehash}{rehash}
    % \SetKwFunction{Append}{append}
    % \SetKwFunction{isEmpty}{isEmpty}
    % \SetKwFunction{isFull}{isFull}
    \SetKwArray{HashList}{HashList}

    \Input{a set of elements $\mathcal{T} =\left\lbrace e_1, e_2, \cdots, e_n \right\rbrace$, number of buckets $m$}
    \Output{a hash table.}

    $M \gets m$

    \ForAll{$e \in \mathcal{T}$}{
        \If{\HashList{\Hash{$j$,$e$}} is full}{
            $j\gets j + 1$

            $M\gets M + m\cdot 2^{j-1}$

            Put $e$ into bucket \HashList{\Hash{$j$, $e$}}

        }
        \Else{Put $e$ into bucket \HashList{\Hash{$j$, $e$}}}
    }
    \Return{\HashList}
\end{algorithm}

\begin{algorithm}[h]
    \label{algo-rehashing2}
    \caption{Rehashing v2}
    \SetKwInOut{Input}{Input}
    \SetKwInOut{Output}{Output}
    \SetKwFunction{Hash}{Hash}
    \SetKwData{True}{True}
    \SetKwData{False}{False}
    % \SetKwFunction{Rehash}{rehash}
    % \SetKwFunction{Append}{append}
    % \SetKwFunction{isEmpty}{isEmpty}
    % \SetKwFunction{isFull}{isFull}
    \SetKwArray{HashList}{HashList}
    \SetKwArray{ListUsed}{ListUsed}

    \Input{a set of elements $\mathcal{T} =\left\lbrace e_1, e_2, \cdots, e_n \right\rbrace$, number of buckets $m$}
    \Output{a hash table.}

    $M \gets m, j\gets 0$


    \ForAll{$e \in \mathcal{T}$}{
        \tcc{Find the last bucket that has been used for current element}

        \While{\ListUsed{\HashList{\Hash{$j$,$e$}}}$=$\False and $j > 0$}{
            $j\gets j - 1$
        }
        \tcc{If the last-used bucket is full then split it}
        \If{\HashList{\Hash{$j$,$e$}} is full}{
            $j\gets j + 1$

            $M\gets M + m\cdot 2^{j-1}$

            Put $e$ into bucket \HashList{\Hash{$j$, $e$}}

            \ListUsed{\HashList{\Hash{$j$, $e$}}} $\gets$ \True

        }
        \Else{
            Put $e$ into bucket \HashList{\Hash{$j$, $e$}}
        }
    }
    \Return{\HashList}
\end{algorithm}

\end{document}