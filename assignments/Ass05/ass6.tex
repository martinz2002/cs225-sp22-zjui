%        -----------------------------
%        | Group Assignment Template |
%        -----------------------------
%         Author: Zhong Tiantian
%         Date Created: Feb 21, 2022
%         Date Modified: Feb 22, 2022
%         Version: 2.0
%
% -------------------------------------------

\documentclass[12pt]{article}
\usepackage[scale=0.8]{geometry}
\usepackage{fontspec}   % 启用自定义字体
\usepackage{fancyhdr}
\usepackage{graphicx}   % 支持图像
\usepackage{hyperref}   % 支持引用(书签引用)
\usepackage{amsmath}    % 支持数学
\usepackage{amsfonts}   % 支持数学字体
\usepackage{amssymb}    % 支持数学符号
\usepackage{enumerate}
\usepackage{bm}
\usepackage{mathptmx}
\usepackage[nofillcomment, linesnumbered,ruled,lined,boxed]{algorithm2e}% 支持伪代码
\usepackage{listings}
\usepackage{fontspec}
\usepackage{booktabs}
\usepackage{tikz}
\usetikzlibrary {graphs, graphdrawing}
\usegdlibrary {trees}

% \everymath{\displaystyle}
\setmainfont{Times New Roman}
% \setmonofont{Consolas}
% \setsansfont{Arial}
\lstset{                  %设置代码块
    basicstyle=\small\fontspec{Consolas},% 基本风格
    numbers=left,    % 行号
    numbersep=10pt,  % 行号间隔
    tabsize=4,       % 缩进
    extendedchars=true, % 扩展符号?
    breaklines=true, % 自动换行
    frame=shadowbox,  % 框架左边竖线
    xleftmargin=19pt,% 竖线左边间距
    showspaces=false,% 空格字符加下划线
    showstringspaces=false,% 字符串中的空格加下划线
    showtabs=false,  % 字符串中的tab加下划线
}
\pagestyle{fancy}
\fancyhf{}
\fancyhead[L]{\textsc\COURSECODE \\ \textbf\HWTITLE}
\fancyhead[C]{}
\fancyhead[R]{Last Modified on \\ \today}
\fancyfoot[L]{\AUTHOR}
\fancyfoot[C]{}
\fancyfoot[R]{--\thepage--}

\fancypagestyle{firstPage}{
    \fancyhf{}
    \fancyfoot[R]{--\thepage--}
}

\newcommand{\E}{\text{e}}
\newcommand{\Diff}{\text{d}}

\title{\textsc\COURSECODE \\ \textbf{\HWTITLE}}
\author{\AUTHOR}
\date{\textit{\normalsize Last Modified on \today}}

\newenvironment{questions}{\begin{enumerate}[Ex. 1]}{\end{enumerate}}
\newenvironment{parts}{\begin{enumerate}[(i)]}{\end{enumerate}}
\newcommand{\question}[1]{\newpage \item \textsc{\textbf{Our Answer.}} (This part is written by #1)\newline}
\renewcommand\part{\item}
\newcommand\makeMyTitle{\maketitle
                        % Please add the following required packages to your document preamble:
% \usepackage{booktabs}
\begin{table}[htp]
    \centering
    \begin{tabular}{@{}clc@{}}
        \toprule
        \multicolumn{2}{c}{\textbf{Members}}                                                               \\ \midrule\midrule
        \multicolumn{1}{c|}{\textbf{Name}}  & \textbf{Student ID} \\ \midrule
        \multicolumn{1}{c|}{Li Rong}        & 3200110523          \\
        \multicolumn{1}{c|}{Zhong Tiantian} & 3200110643          \\
        \multicolumn{1}{c|}{Zhou Ruidi}     & 3200111303          \\
        \multicolumn{1}{c|}{Jiang Wenhan}   & 3200111016          \\ \bottomrule
    \end{tabular}
\end{table} 
                        \vspace{5cm}
                        \centering\textsc{\Large{Please turn over for our answers.}}\normalsize}

\renewcommand{\baselinestretch}{1.2}
% \setlength{\parskip}{0ex}



\def\HWTITLE{Homework 5}
\def\COURSECODE{CS 225: Data Structures}
\def\AUTHOR{Group D1}


\begin{document}
\makeMyTitle
\thispagestyle{firstPage}

\begin{questions}

    \question{Zhong Tiantian}
    We assume that the list \texttt{next[]}, the so-called ``prefix table'', satisfy the following properties:
    \begin{enumerate}
        \item $\texttt{next[0]} = 0$, since the first element has no prefixes.
        \item \texttt{next[$i$] $= t$} indicates that the $i$-th element in \texttt{pList[]} is equal to the $t$-nd one.
    \end{enumerate}

    Assuming that we have known \texttt{next[0], next[1], ..., next[$x-1$]}, and denote \texttt{next[$x-1$]} as $ptr$:

    \begin{enumerate}
        \item If \texttt{pList[$x$] $=$ pList[$ptr$]}, which indicates the $ptr$-th element in \texttt{pList[]} is equal to the $x$-th, then extend the current sublist: \texttt{next[$x$]:=next[$x-1$]}.

        \item If \texttt{pList[$x$] $\neq$ pList[$ptr$]}, then go backward by reducing $ptr$ to \texttt{next[$ptr-1$]}.
    \end{enumerate}

    The algorithm is defined below.

    \begin{algorithm}
        \caption{Build \texttt{Next[]}}
        \label{alg-build-next}
        \SetKwArray{pList}{pList}
        \SetKwData{Next}{next}
        \SetKwData{Now}{ptr}
        \SetKwFunction{Append}{append}
        \SetKwFunction{Length}{getLength}
        \SetKw{Call}{call}

        \Next.\Append{$0$}

        $x \gets 1$

        $\Now \gets 0$

        \While{$x < \Length{\pList}$}{
            \If{$\pList{\Now} = \pList{x}$}{
                $\Now \gets \Now + 1$

                $x \gets x + 1$

                \Call \Next.\Append{\Now}

            }
            \ElseIf{$\Now \neq 0$}{\Now $\gets$ \Next{$\Now - 1$}}
            \Else{
                \Call \Next.\Append{$0$}

                $x \gets x + 1$
            }

        }
    \end{algorithm}

    \paragraph{Time Complexity Analysis}
    In most cases, the algorithm iterate one by one with $x\gets x + 1$, so it will perform \texttt{Length(pList)$=m$} times; in relatively rare cases, where $ptr$ goes backward, the extra time spent can be amortised to the general cases. Thus the amortised time complexity is $O(\mbox{\texttt{length(pList)}}) = O(m)$.

    \question{Jiang Wenhan}
    \paragraph{Algorithm Description}Assume the length of pattern sequence is $m$.
    \begin{itemize}
        \item Calculate the next sequence of the pattern sequence, which takes time $O(m)$, $m<n$ and align $P[0]$ and $S[0]$.
        \item Begin comparison between $P[0]$ and $S[0]$, if $P[0]\neq S[0]$.
        \item Turn back to $S[1]$, and keep doing things above until we found $P[0]=S[k], k\leq n$ ,

              \textbf{Once we found such $k$,}
              \begin{enumerate}
                  \item Align $P[0] $ with $S[k]$ and keep comparing them until we found $P[a]\neq S[k+a]$.
                  \item Right shift the pattern sequence for $m-next[a]$ where front maximal common sublist overlap with the original position of back maximal common sublist.
                  \item Repeat the comparison at this new position (begin with $P[next[a]]$ and $S[k+a]$).
              \end{enumerate}

              \textbf{Whenever we found that $a = m$,}
              \begin{enumerate}
                  \item Record the position $k$ , where the pattern sequence has occur.
                  \item Right shift the pattern sequence by $m-next[m]$ and start new comparison at $P[next[m]]$ and $S[k+m]$.
                  \item Repeat doing this until we reach the end, where the $P[m-1]$ is aligned with $S[n-1]$.
              \end{enumerate}
    \end{itemize}

    \paragraph{Time Complexity Analysis}To calculates its time complexity, we make an assumption of the worst case, where the pattern sequence and target sequence both consists of pairwise different elements. In such case, all elements in $next$ sequence are all $0$ . The times of shift is $\frac{n}{m}$, and the comparisons for each shift is $m$ , so the time complexity for comparison is $m\times\frac{n}{m} $, which is $n$. The total time complexity is $O(m)+n$ , since m is less equal than n, so the total time complexity is $O(n)$.

    \question{Zhou Ruidi}
    \begin{parts}
        \part
        \textbf{Method 1} (Might not be correct; see \textbf{Method 2} for a better solution)
        
        Here is the algorithm.

        \begin{algorithm}
            \caption{Multiplying two polynomials.}
            \SetKwData{List}{list}

            \tcc{From $n_1$ to $n_d$ we choose one by one.}
            \tcc{Denote the length of the list as $d$.}

            \ForAll(calculate each number from $n_1$ to $n_d$){$n_i\in \List$ and $1\leq i \leq d$}{


                \For{$j = 2, 3, \cdots, d$}{

                    \tcc{calculate $P(n_i)$}

                    \For{$m = 2, 3, \cdots, j$}{
                        \tcc{figure out each element in the polynomial}

                        $n_i\gets n_i \cdot n_i$

                        \tcc{record this number, $k$ iterates from 2 to d, and we know that $a_d = 1$}

                        $P_k \gets a_k \cdot n_i$
                    }
                }
            }
        \end{algorithm}

        Calculate $P(n_i)$ by adding up the $P_k+a_0+a_1\cdot n_i$, and finally each $P(n_i)=0$. After that we can calculate the unknown coefficients $a_0, a_1, \cdots, a_{d-1}$. And the time complexity is $O(d\cdot d \cdot(d+1)/2) \approx O(d^3)$.

\vspace{1.5cm}

        \textbf{Method 2}

        For monic polynomial, it has the form like: $P(x)=(x-n_1) \cdot (x-n_2)\cdots (x-n_d)$  (because $a_d=1$) .

        We pair two elements to make one pair and we will get $d/2$ pairs: $(x-n_1)\cdot(x-n_2), (x-n_3)\cdot(x-n_4)$... and the time complexity for calculate each pair is $O(1)$ from the function before we know (Multiplying polynomials of degree $i$ and degree 1 has time complexity $O(i)$).

        Then we pair these $d/2$ polynomials to get $d/4$ polynomials and for each polynomial we spend $2\log 2$ using the function above (Degree-$i$ polynomial multiplying the degree-$i$ need $O(i\log i)$ . So for this step we need $(d/4)\cdot(2\log 2)$

        Repeat the above steps until getting the answer by multiplying two polynomials or three polynomials. The total time we need is
        \begin{align*}
            \sum_{i} \frac{d}{2^i} \cdot 2^{i-1} \cdot \log 2^{i-1} & = \frac{d}{2}(1+\log 2+\log 4+...) \\
                                                                    & =\frac{d}{2}(\log d)^2             \\
                                                                    & \approx O(d\log^2 d)
        \end{align*}

        \part
        The algorithm we designed in the previous part has satisfied the requirement of time complexity $O(n\log^2 n)$.

        Now let's prove the functions mentioned above are elementary.

        Obviously, the time complexity of multiplying a degree-1 polynomial with a degree-$i$ polynomial is $O(i)$.

        Let's consider multiplying two degree-$i$ polynomials:
        \begin{enumerate}[(1)]
            \item Suppose $P(x)=\left[ x_0,x_1,x_2, \cdots, x_i \right], P^\prime(x)=\left[ x^\prime_0,x^\prime_1,x^\prime_2, \cdots, x^\prime_i \right]$ (assuming $i$ is the power of 2; yet things will remain similar if $i$ is not).
            
            \item Perform $FFT(P(f_x)), FFT(P(f_y))$, where
            \[
                P(f_x)=\left[ x_0,x_1,x_2, \cdots, x_i,0,0,0, \cdots \right]             
            \]
            and
            \[
                P^\prime(f_x)=[x'_0,x'_1,x'_2, \cdots, x'_i,0,0,0,\cdots]
            \]
            
            (NOTE: The numbers of ending 0-s in the above lists are $i$, respectively).
            
            \item Denote \[ W=e^{2\cdot \pi m/2n}=e^{\pi m/n} \] and thus we get \[ L_1\equiv \left[ P(W^0), \cdots, P(W^{2i-1}) \right] \] and \[ L_2 \equiv \left[ P'(W^0)\cdots P'(W^{2i-1}) \right] \].
            
            \item Let $L_3 \equiv L_1 \cdot \L_2 \left[ P(W^0)\cdot P'(W^0)\cdots P(W^{2i-1})P'(W^{2i-1})\right]$.
            
            \item Perform $IFFT(L_3)$, which takes $O(i\log i)$.
            
            \item The time complexity: $O(2\cdot 2i\cdot \log 2i +2i+2i\cdot \log 2i)=O(i\log i)$.
        
        \end{enumerate}

        Therefore, the operations are all elementary.

    \end{parts}
\end{questions}

\end{document}