%        -----------------------------
%        | Group Assignment Template |
%        -----------------------------
%         Author: Zhong Tiantian
%         Date Created: Feb 21, 2022
%         Date Modified: Feb 22, 2022
%         Version: 2.0
%
% -------------------------------------------

\documentclass[12pt]{article}
\usepackage[scale=0.8]{geometry}
\usepackage{fontspec}   % 启用自定义字体
\usepackage{fancyhdr}
\usepackage{graphicx}   % 支持图像
\usepackage{hyperref}   % 支持引用(书签引用)
\usepackage{amsmath}    % 支持数学
\usepackage{amsfonts}   % 支持数学字体
\usepackage{amssymb}    % 支持数学符号
\usepackage{enumerate}
\usepackage{bm}
\usepackage{mathptmx}
\usepackage[nofillcomment, linesnumbered,ruled,lined,boxed]{algorithm2e}% 支持伪代码
\usepackage{listings}
\usepackage{fontspec}
\usepackage{booktabs}
\usepackage{tikz}
\usetikzlibrary {graphs, graphdrawing}
\usegdlibrary {trees}

% \everymath{\displaystyle}
\setmainfont{Times New Roman}
% \setmonofont{Consolas}
% \setsansfont{Arial}
\lstset{                  %设置代码块
    basicstyle=\small\fontspec{Consolas},% 基本风格
    numbers=left,    % 行号
    numbersep=10pt,  % 行号间隔
    tabsize=4,       % 缩进
    extendedchars=true, % 扩展符号?
    breaklines=true, % 自动换行
    frame=shadowbox,  % 框架左边竖线
    xleftmargin=19pt,% 竖线左边间距
    showspaces=false,% 空格字符加下划线
    showstringspaces=false,% 字符串中的空格加下划线
    showtabs=false,  % 字符串中的tab加下划线
}
\pagestyle{fancy}
\fancyhf{}
\fancyhead[L]{\textsc\COURSECODE \\ \textbf\HWTITLE}
\fancyhead[C]{}
\fancyhead[R]{Last Modified on \\ \today}
\fancyfoot[L]{\AUTHOR}
\fancyfoot[C]{}
\fancyfoot[R]{--\thepage--}

\fancypagestyle{firstPage}{
    \fancyhf{}
    \fancyfoot[R]{--\thepage--}
}

\newcommand{\E}{\text{e}}
\newcommand{\Diff}{\text{d}}

\title{\textsc\COURSECODE \\ \textbf{\HWTITLE}}
\author{\AUTHOR}
\date{\textit{\normalsize Last Modified on \today}}

\newenvironment{questions}{\begin{enumerate}[Ex. 1]}{\end{enumerate}}
\newenvironment{parts}{\begin{enumerate}[(i)]}{\end{enumerate}}
\newcommand{\question}[1]{\newpage \item \textsc{\textbf{Our Answer.}} (This part is written by #1)\newline}
\renewcommand\part{\item}
\newcommand\makeMyTitle{\maketitle
                        % Please add the following required packages to your document preamble:
% \usepackage{booktabs}
\begin{table}[htp]
    \centering
    \begin{tabular}{@{}clc@{}}
        \toprule
        \multicolumn{2}{c}{\textbf{Members}}                                                               \\ \midrule\midrule
        \multicolumn{1}{c|}{\textbf{Name}}  & \textbf{Student ID} \\ \midrule
        \multicolumn{1}{c|}{Li Rong}        & 3200110523          \\
        \multicolumn{1}{c|}{Zhong Tiantian} & 3200110643          \\
        \multicolumn{1}{c|}{Zhou Ruidi}     & 3200111303          \\
        \multicolumn{1}{c|}{Jiang Wenhan}   & 3200111016          \\ \bottomrule
    \end{tabular}
\end{table} 
                        \vspace{5cm}
                        \centering\textsc{\Large{Please turn over for our answers.}}\normalsize}

\renewcommand{\baselinestretch}{1.2}
% \setlength{\parskip}{0ex}



\def\HWTITLE{Homework 3}
\def\COURSECODE{CS 225: Data Structures}
\def\AUTHOR{Group D1}


\begin{document}
\makeMyTitle
\thispagestyle{firstPage}

\begin{questions}
    \question
    \begin{parts}
        \part
        In order to gain the smallest number in the $n$ numbers, At first, suppose we know the exact number which is the smallest number, we should compare it with the other $(n-1)$ numbers which means we need $(n-1)$ comparisons to ensure it is the smallest number. For the specific method, we could select two numbers randomly and then compare them. Then we choose the smaller number and compare it with another number in the left $(n-2)$ numbers. Then repeat this step. We need $(n-1)$ steps (comparisons) to know the smallest number.

        \part
        Actually, when we know from the first question, we should use $(n-1)$ comparisons to know the smallest numbers, and then we could know the second smallest number after this step. In fact, in the first question, we have already compared the smallest number with the second smallest number, we can make use of this hidden condition to decrease the comparison times to find the second smallest number after having found the smallest number in the list. We can only find the second smallest number in the number list which has been compared with the smallest number in the first step.

        From Figure \ref{fig-FakeTree} we know that: At first, we separate the numbers in pairs and compare them (no matter the number is odd or even), Then we can get the smaller number in each pair and we will get the smallest number at last which needs $(n-1)$ comparisons in total. Then we just find the number of elements which have been compared with the smallest number, we can know that there is only one number in each layer(The number which are yellow). For $n$ numbers, there are $\log n$ levels in total, so there are $\log n$ elements which have been compared with the smallest number. Through the first question, we know we need $(\log n - 1)$ comparisons to know the smallest number in the $\log n$ elements (the smallest number in these elements is the second smallest number in the whole list), So we need $(n-2+\log n)$ comparisons\footnote{We suspect that there's a typo in the assignment sheet, which said the number of comparisons should be $(n-1+\log n)$, one comparison larger than ours.} to know the smallest number and the second smallest number.

        \begin{figure}[h]
            \centering

            \label{fig-FakeTree}
            \tikz [grow' = up, tree layout, sibling distance=8mm, mark/.style={fill=violet, text=white}, mark/.default=, markmin/.style={fill=red, text=white}, markmin/.default=]
            \graph [nodes={circle, draw, inner sep=1.5pt}]{ [fresh nodes]

            1[markmin] <- {{2 [mark]} <- {2 <- {2, 5}, 3 <- {3, 8}}, 1 <- {4[mark] <- {4, 6}, 1 <- {3[mark], 1}}}

            };
            \caption{An example structure for the algorithm.}
        \end{figure}

        \part
        % \begin{algorithm}[h]
        %     \SetKwData{AList}{AList1}
        %     \SetKwData{AListLength}{AListLength}
        %     \SetKwFunction{Pairs}{PAIRS}

        %     \tcp{The number of elements in \AList is \AListLength.}

        %     $n \gets \frac{n}{2}$


        % \end{algorithm}

        Here is a Python implement of our algorithm. For detailed explanations, see the comments.

        \lstinputlisting[language=python]{ex1.py}

    \end{parts}

    \question
    The algorithm is to keep two memory positions for the current-largest
    element and current-smallest element. For the first element, there is no
    comparison. For the second element, there is only one comparison with
    the first element.

    Then, for the left \((n-2)\) elements including the third element. There
    are only two comparisons for one element, comparison with the
    current-largest element and current-smallest element. If it is larger
    than the current-largest, it will take its place and then the algorithm
    goes for next element. If it is smaller than the current-smallest, then
    it will take its place and then the algorithm goes for next element.
    Otherwise it is ignored.

    Therefore, the count for total comparison is \(2(n-2)+1\) ,which is
    \(2n-3\) .

    \begin{parts}
        \part
        For the best situation, all the numbers are arranged in the sequence
        from small to large. Then for every element, there is only one
        comparison needed. The overall comparison is \(n-1\) . For the worst
        situation, all the numbers are arranged in the sequence from large to
        small. The overall comparison is \(2n-3\) .
        \part
        The number of comparison will not change whether \(n\) is a power of 2
        or not.
    \end{parts}

    \question
    The first step is to create a library that contains \(n\) tags that represent every element in the totally ordered set. Besides, each tag also contains a counter that records the times each time element has been found.

    The next thing we need to do is to traverse the list, each time we find a elements, we add one to the corresponding counter. After the list traverse, we traverse the library to find which element is majority.

    The structured algorithm is described in Algorithm \ref{alg-majority}

    \begin{algorithm}
        \caption{Majority Element}
        \label{algo-majority}
        \SetKwArray{Lib}{library}
        \SetKwData{tags}{tag}
        \SetKwData{isMajor}{isMajor}
        \SetKwData{True}{True}
        \SetKwData{False}{False}
        \SetKwFunction{CreateLib}{CreateLibrary}
        \SetKwInOut{Input}{Input}
        \SetKwInOut{Output}{Output}

        \Input{Two lists $T$, $L$}

        \Output{\isMajor, denoting whether $L$ contains any major elements.}

        $\isMajor \gets \False$

        \tcc{First create libraries, or "buckets", for each unique element in $T$.}4

        \ForEach{$t \in T$}{
            \If{$t \notin \Lib$}{\CreateLib{$t_i$}}
        }

        \ForEach{$l_i\in L$}{
            \Lib{$l_i$} $\gets$ \Lib{$l_i$} $+ 1$
        }

        \ForEach{\Lib{$i$}}{
            \If{\Lib{i} > $n/2$}{$\isMajor\gets \True$}
        }
        \Return{\isMajor}
    \end{algorithm}



\end{questions}


\end{document}