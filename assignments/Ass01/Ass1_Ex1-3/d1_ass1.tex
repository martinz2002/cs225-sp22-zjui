% -------------------------------------------
%          请使用XeTeX, XeLaTeX编译
%        -----------------------------
%        | Group Assignment Template |
%        -----------------------------
%         Author: Zhong Tiantian
%         Date Created: Feb 21, 2022
%         Date Modified: Feb 22, 2022
%         
% -------------------------------------------
%
%

\documentclass[answers]{exam}
% \usepackage{ctex}   % 引入中文支持
\usepackage{graphicx}   % 支持图像
\usepackage{hyperref}   % 支持引用(书签引用)
\usepackage{amsmath}    % 支持数学
\usepackage{amsfonts}   % 支持数学字体
\usepackage{amssymb}    % 支持数学符号
\usepackage[linesnumbered,ruled,lined,boxed,commentsnumbered]{algorithm2e}% 支持伪代码

% --- set math font ---
\usepackage{mathptm}   % 设置数学公式字体为 Adobe Times
\usepackage{bm}         % 在数学公式中启用粗体
% ---------------------
\usepackage{booktabs}   % 启用书签
% \usepackage[scale=0.8]{geometry}    % 设置页边距
\usepackage{fontspec}   % 启用自定义字体

% \unframedsolutions  % 设置Solution不带边框
\setmainfont{Times New Roman} % 设置正文字体为 Times New Roman

% ------ 页面设置 ------

\setlength\linefillheight{.5in} % 行间距
\pagestyle{headandfoot} % 引入页眉和页脚
\firstpageheadrule  % 首页页眉横线
\firstpagefootrule
\runningheadrule    %正文页页眉横线
\runningfootrule
\firstpagefooter{\large{D1}}{}{--\thepage--}    % 首页页脚
\runningfooter{\large{D1}}{}{--\thepage--}  % 正文页页脚

\renewenvironment{solution}{\textsc{\textbf{Solution:}} \par}{}


% ------ !!!!填写作业标题!!!!
\firstpageheader{ZJUI CS225 D1}{\huge{\textbf{Homework 1}}\\ \small{Jiang Wenhan, Li Rong, Zhong Tiantian, Zhou Ruidi}}{Last Modified:\\ \today}   % 首页页眉{左}{中}{右}
\runningheader{CS225\\Homework 1}{}{Last Modified:\\ \today}  % 正文页眉
% ------ !!!!填写作业标题!!!!

\begin{document}

\begin{questions}

	\question
	% Question 1 solved by Tiantian.
	\begin{solution}
		% 在这里写第一题的解答。
		We suppose the list $L$ has members: $length$ denoting the length of the list, or number of elements in the list; $data[i]$ denoting the $i$-th element of the list. Assuming the index starts from 1.

		Therefore, just cut down the length of $l$ to disable elements after the $i$-th in the list and w obtain $L.length := L.length-k$,
		so the elements $data[i+1, i+2, \cdots, length]$ will be "ignored", or deleted.

		Since this only requires one instruction thus the complexity has nothing to do with $k$, the answer should be $\Theta(1)$.

	\end{solution}

	\question
	\begin{parts}
		\part
		\begin{solution}
			% 第二题(i)的解答
			Solution to 2(i)
		\end{solution}
		\part
		\begin{solution}
			% 第二题(ii)的解答
			Solution to 2(ii)
		\end{solution}

		\part
		\begin{solution}
			% 第二题(iii)的解答
			Solution to 2(iii)
		\end{solution}
	\end{parts}

	\question
	\begin{solution}
		% 第三题
	\end{solution}



	% 第四大题
	\question
	\begin{solution}
		Please refer to the code bundle submitted.
	\end{solution}
\end{questions}
\end{document}